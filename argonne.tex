\documentclass{aip-cp}

\usepackage[numbers]{natbib}
\usepackage{rotating}
\usepackage{graphicx}
\usepackage[dvipsnames]{xcolor}

\newif\iffinal
% Un-comment this line to see proposal without comments
%\finaltrue

\iffinal
    \newcommand\ian[1]{}
\else
    \newcommand\ian[1]{{\color{red}[Ian: #1]}}
\fi


% Document starts
\begin{document}

% Title portion
\title{Data Automation at Light Sources:\\Experiments and Lessons Learned\ian{more exciting title needed}}

\author[aff1,aff2]{Author's Name\corref{cor1}}
%\eaddress[url]{address@domain1.edu}
\author[aff2]{Author's Name}
%\eaddress{anotherauthor@thisaddress.yyy}

\affil[aff1]{Data Science and Learning Division, Argonne National Laboratory, Argonne IL 60439, USA.}
\affil[aff2]{Department of Computer Science, University of Chicago, Chicago IL 60637, USA.}
%\affil[aff3]{You would list an author's second affiliation here.}
\corresp[cor1]{Corresponding author: foster@anl.gov}
%\authornote[note1]{This is an example of first authornote.}
%\authornote[note2]{This is an example of second authornote.}

\maketitle


\begin{abstract}
%The AIP Proceedings article template has many predefined paragraph styles for you to use/apply as you write your paper. To format your abstract, use the \LaTeX template style: {\itshape Abstract.} Each paper must include an abstract. Begin the abstract with the word ``Abstract'' followed by a period in bold font, and then continue with a normal 9 point font.
Rapidly growing data volumes at light sources demand increasingly automated data collection, distribution, and analysis processes, in order to enable new scientific discoveries while not overwhelming finite human capabilities. I present here three projects that use cloud-hosted data automation and enrichment services, institutional computing resources, and high- performance computing facilities to provide cost-effective, scalable, and reliable implementations of such processes. In the first, Globus cloud-hosted data automation services are used to implement data capture, distribution, and analysis workflows for Advanced Photon Source and Advanced Light Source beamlines, leveraging institutional storage and computing. In the second, such services are combined with cloud-hosted data indexing and institutional storage to create a collaborative data publication, indexing, and discovery service, the Materials Data Facility (MDF), built to support a host of informatics applications in materials science. The third integrates components of the previous two projects with machine learning capabilities provided by the Deep Learning Hub (DLHub) to enable on-demand access to machine learning models from light source data capture and analysis workflows, and provides simplified interfaces to train new models on data from sources such as MDF on leadership scale computing resources. I draw conclusions about best practices for building next-generation data automation systems for future light sources.
\end{abstract}


% Head 1
\section{INTRODUCTION}

\ian{Potential authors: Ben Blaiszik, Kyle Chard, Ryan Chard, Logan Ward, Justin Wozniak, ...}

TBD.


References: MDF~\cite{MDF2016}, networking materials data~\cite{foster2015networking}, Justin~\cite{wozniak2015big}.


\section{DATA AUTOMATION}

Figures from my 

Choose next experiment.


\section{PRACTICAL EXPERIENCES}

\subsection{Early Work}

We first experimented with online analysis of APS in the late 1990s ...
computing microtomography~\cite{wang1999quasi,wang2001high} and with Westbrook~\cite{von2000using}.

\subsection{Current Status}



\section{FUTURE}






% Figure
%\begin{figure}[h]
%  \centerline{\includegraphics[width=150pt]{art/fig_1}}
%  \caption{To format a figure caption use the \LaTeX template style: Figure Caption. The text ``FIGURE 1,'' which labels the caption, should be bold and in upper case. If figures have more than one part, each part should be labeled (a), (b), etc. Using a table, as in the above example, helps you control the layout.}
%\end{figure}

%\begin{sidewaysfigure}
%  \centerline{\includegraphics[width=500pt]{art/fig_2}}
%  \caption{To format a figure caption use the \LaTeX template style: Figure Caption. The text ``FIGURE 2,'' which labels the caption, should be bold and in upper case. If figures have more than one part, each part should be labeled (a), (b), etc. Using a table, as in the above example, helps you control the layout.}
%\end{sidewaysfigure}



\section{SUMMARY}




% Sections that will go in second font

% Acknowledgement
\section{ACKNOWLEDGMENTS}

This work was supported in part by DOE contract DE-AC02-06CH11357.

% References

\nocite{*}
\bibliographystyle{aipnum-cp}%
\bibliography{Bibs/refs}%


\end{document}
